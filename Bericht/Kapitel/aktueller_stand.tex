\section{Aktueller Stand}
\subsection{}

\section{Geplant}
\subsection{Software}
Das Skript zum Auswerten der Daten soll eine Benutzeroberfläche bekommen. Diese ermöglicht dem Nutzer für die angeschlossenen Module Funktionen zuzuweisen, zum Beispiel einer Taste auf einer Tastatur eine Tastenkombination zu legen. Diese Zuweisungen werden dann in einem dict-Format gespeichert und in einer JSON Datei abgelegt, damit die Einstellungen bei Neustart des PCs noch vorhanden sind. \\\\
Fehlererkennung bei der Datenübertragung. Ähnlich wie bei CAN oder USB soll nach der Kommunikation zwischen zwei Teilnehmern eine Fehlererkennung laufen und bei einer fehlerhaften Nachricht diese ignoriert werden. \\\\

\subsection{Generell}
- Bestellen der PCBs und Zusammenbau eines finalen Showprototyps mit funktionierender Modularität. Aufbau von mehreren Tastatur-Modulen, um diese auch zu testen und vorzuzeigen. \\
- Gehäusedesign. Im Zuge des ersten Sprints ist bereits ein grobes Design entwickelt worden. Mit Abschluss des PCB Designs kann nun auch ein Gehäuse entwickelt und gedruckt werden. \\
- Entwicklung eines Audiomoduls mit Potentiometern und einem Fader. Durch die verallgemeinerte Entwicklung des ersten Moduls sollte die Entwicklung weiterer Module relativ einfach sein, da sich lediglich mit der Funktion des Moduls auseinandergesetzt werden muss. Der Kommunikation zwischen dem Modul-Mikrocontroller und dem Hauptmodul ist im Grunde die gleiche.