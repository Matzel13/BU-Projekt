\chapter{Einleitung}
Moderne Eingabegeräte sind ein essenzieller Bestandteil des digitalen Zeitalters und finden in verschiedensten Bereichen wie der Datenverarbeitung, Musikproduktion und erst recht in der industriellen Steuerung Anwendung. Die Anforderungen an diese Geräte können dabei sehr vielseitig sein. Um diesen Anforderungen gerecht zu werden, befasst sich das Projekt mit dem Ziel, eine modulares Eingabegeräte-System zu entwickeln.
\\
\\
Die Modularität ermöglicht es, verschiedene Module wie z.B. ein Keypad-Modul mit einem 4x4-Tastenraster oder ein Audiomodul mit Potenziometern und Pegelstellern in einem erweiterbaren System zu kombinieren. Dies bietet die Möglichkeit, das System entsprechend der individuellen Vorstellung zu erweitern.
\\
\\
Im Mittelpunkt des Projekts steht die Entwicklung eines speziell entworfenen Bussystems, das eine Kommunikation zwischen den einzelnen Modulen ermöglicht. Zu diesem Datenbus gehört die Umsetzung der Bitübertragung, die Adressierung und Skalierbarkeit der Module, sowie Timing und Synchronisation. Ziel ist es, einen robusten und effizienten Datenbus zu schaffen, der eine hohe Busgeschwindigkeit sowie fehlerfreie Übertragungen gewährleistet.
\\
\\
Ergänzend dazu umfasst das Projekt die Auswahl und Validierung geeigneter elektronischer Komponenten, die Optimierung des PCB-Designs sowie die Entwicklung eines Gehäuses, das die Modularität des Systems unterstützt. Darüber hinaus wird ein Übersetzerprogramm entwickelt, das die Daten zwischen den Modulen und des benutzten Computers verarbeitet. Der Benutzer wird, in der die Möglichkeit bekommen, die Tasten nach Belieben zu belegen.
\\
\\
Das Ziel ist eine vielseitige und individuell anpassbare Lösung für Anwender, insbesondere im Bereich von Programmen wie den Adobe-Bearbeitungstools, die zahlreiche wichtige Tastenkombinationen erfordern.
\newpage