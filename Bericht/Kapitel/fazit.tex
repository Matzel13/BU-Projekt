\section{Fazit}
Die agile Herangehensweise stellt  Herausforderungen dar, insbesondere wenn bisher wenig praktische Erfahrung mit dieser Art der Projektleitung vorhanden ist. Dennoch konnten klare Vorteile der agilen Entwicklung festgestellt werden. So wurde beispielsweise das anfängliche Ziel, mehrere Module zu entwickeln, rechtzeitig angepasst, um den zeitlichen Einschränkungen gerecht zu werden. Darüber hinaus war es möglich, Aufgaben gemeinsam zu bearbeiten oder neu zuzuweisen, wenn einzelne Gruppenmitglieder ausfielen.

Es zeigt sich jedoch, dass eine gründlichere Planungsphase dazu hätte beitragen können, einige Probleme frühzeitig zu erkennen und zu vermeiden. So haben beispielsweise die differenzielle Datenübertragung und das Flashen der ATtinys und ATmegas viel Zeit in Anspruch genommen. Schwierigkeiten, die vermutlich durch eine bessere Abstimmung der Komponenten im Vorfeld hätten reduziert oder ganz vermieden werden können.

Des Weiteren wird in Anbetracht der noch verbleibenden Zeit das Audio-Modul aus dem Plan für dieses Projekt gestrichen und sich erstmal lediglich auf das Tastaturmodul fokussiert. Da dieses Projekt nach Abschluss des Faches \glqq Bussysteme und Sensorik\grqq{} für den privaten Gebrauch der Gruppenmitglieder weiterentwickelt wird, wird dieses Modul wahrscheinlich in Zukunft eingeführt. Das grobe Layout eines weiteren Moduls wird sich nicht grundlegend von dem des Tastaturmoduls unterscheiden, daher liegt die Priorität erstmal in der Optimierung des vorhandenen Moduls.

\newpage
\subsection{Persönliches Fazit}
\subsubsection{Johanna Boettcher}
Aus dem Projekt habe ich sehr viel mitnehmen können. Für mich persönlich war eine der größten Herausforderungen, die Umsetzung der differenziellen Übertragung im Layout, vor allem die Schwierigkeit trotz der vielen Jumper Verbindungen die Leiterbahnen nah beieinander und gleichlang zu halten. Doch wir konnten sehr viel aus genau diesen Schwierigkeiten für unseren zweiten Prototypen lernen. Ich bin sehr glücklich mit dem Ergebnis, da wir einen sehr guten Eindruck haben, was verbessert werden kann und wie die nächsten Schritte aussehen. 
\newline
Im Rückblick haben wir den Zeitaufwand für das Projekt etwas unterschätzt. Gerade durch unerwartete Herausforderungen wie Lieferschwierigkeiten und das Einarbeiten in neue Konzepte, war es schwierig, den Zeitrahmen einzuhalten. Dennoch war es eine wertvolle Erfahrung, die vor allem bei der Gestaltung und der Planung der nächsten Platine sehr wertvoll sein wird.

\subsubsection{Marcel Eßmann}
Das Projekt hat mir sehr viel Freude bereitet. Da wir alle von der Idee überzeugt waren, ließ sich recht schnell ein Plan entwickeln. Die Probleme, die wir hatten konnten wir meist mit gemeinsamer Denkleistung beheben oder alternativen finden.
Wir haben uns vermutlich etwas viel vorgenommen, da wir jetzt am Ende eigentliche in komplett fertiges Produkt haben und nutzen wollten. Dies hat jedoch dazu geführt, dass wir als Gruppe auch nach Abschluss des Faches \glqq Bussysteme und Sensorik\grqq{} noch weiter an diesem Produkt arbeiten werden und zumindest ein für uns zufriedenstellendes Produkt entwickeln. 
Durch das Projekt konnten wir einen guten Einblick in den Ablauf der Prototypenentwicklung bekommen und feststellen, dass nicht immer alles rund läuft.

\subsubsection{Peter Fischer}
Auch wenn die Entwicklung nicht immer glattlief und wir uns zu viel für diesen Zeitraum vorgenommen haben, hat mir das Projekt wirklich Spaß gemacht. Ich fand es gut, dass man eine Gruppe finden musste, die sich auf ein gemeinsames Thema einigt und jeder seine Vorstellungen und Ideen einbringen kann. Ich bin nicht enttäuscht, dass unsere Vorstellung nicht ganz erfüllt werden konnte, sondern eher erfreut, was wir erreicht haben. Am Anfang des Semesters hatte keiner von uns eine Idee, wie wir einen Datenbus selber schreiben, eine Platine für ein Keypad entwickeln oder unsere Hardware Befehle auf dem Computer ausführen lassen. Das alles mussten wir uns teilweise frühzeitig selbständig erarbeiten und ein Protokoll entwerfen, über das wir mit dem Computer kommunizieren. Hätten wir gewusst, was alles auf uns zukommt, dann hätten wir unsere Vorstellung von vorne rein nicht so voll gestaltet. Nichtsdestotrotz, bin ich sehr froh über unseren funktionierenden Prototypen und möchte ihn weiterentwickeln, damit ich das \glqq modulare Eingabesystem\grqq{} in voller Form selber nutzen kann.

\subsubsection{Jannik Wendt}
Durch dieses Projekt konnte ich einen Einblick in die Komplexität der Buskommunikation erhalten. Von der Hardware, die nötig ist, um eine differenzielle Übertragung zu ermöglichen, über die Implementierung der Software und dem langwierigen Debuggen dieser mithilfe eines Oszilloskops. \\
Nur durch die Aufgabenteilung und der schlussendlichen Zusammenführung der Einzelkomponenten war es möglich dieses Projekt so weit voranzutreiben und zu einem Stand zu bringen, mit dem eine grundlegende Funktion möglich ist.\\ 
Die Weiterführung dieses Projekts wird hoffentlich zu einem Produkt führen, welches wir tatsächlich im Alltag nutzen können und werden.