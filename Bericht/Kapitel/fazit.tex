\section{Fazit}
Die agile Herangehensweise stellt  Herausforderungen dar, insbesondere wenn bisher wenig praktische Erfahrung mit dieser Art der Projektleitung vorhanden ist. Dennoch konnten klare Vorteile der agilen Entwicklung festgestellt werden. So wurde beispielsweise das anfängliche Ziel, mehrere Module zu entwickeln, rechtzeitig angepasst, um den zeitlichen Einschränkungen gerecht zu werden. Darüber hinaus war es möglich, Aufgaben gemeinsam zu bearbeiten oder neu zuzuweisen, wenn einzelne Gruppenmitglieder ausfielen.

Es zeigt sich jedoch, dass eine gründlichere Planungsphase dazu hätte beitragen können, einige Probleme frühzeitig zu erkennen und zu vermeiden. So haben beispielsweise die differenzielle Datenübertragung und das Flashen der ATtinys und ATmegas viel Zeit in Anspruch genommen. Schwierigkeiten, die vermutlich durch eine bessere Abstimmung der Komponenten im Vorfeld hätten reduziert oder ganz vermieden werden können.

Des Weiteren wird in Anbetracht der noch verbleibenden Zeit das Audio-Modul aus dem Plan für dieses Projekt gestrichen und sich erstmal lediglich auf das Tastaturmodul fokussiert. Da dieses Projekt nach Abschluss des Faches \glqq Bussysteme und Sensorik\grqq{} für den privaten Gebrauch der Gruppenmitglieder weiterentwickelt wird, wird dieses Modul wahrscheinlich in Zukunft eingeführt. Das grobe Layout eines weiteren Moduls wird sich nicht grundlegend von dem des Tastaturmoduls unterscheiden, daher liegt die Priorität erstmal in der Optimierung des vorhandenen Moduls.