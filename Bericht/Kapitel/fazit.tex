\section{Fazit}
Die agile Herangehensweise stellt  Herausforderungen dar, insbesondere wenn bisher wenig praktische Erfahrung mit dieser Art der Projektleitung vorhanden ist. Dennoch konnten klare Vorteile der agilen Entwicklung festgestellt werden. So wurde beispielsweise das anfängliche Ziel, mehrere Module zu entwickeln, rechtzeitig angepasst, um den zeitlichen Einschränkungen gerecht zu werden. Darüber hinaus war es möglich, Aufgaben gemeinsam zu bearbeiten oder neu zuzuweisen, wenn einzelne Gruppenmitglieder ausfielen.

Es zeigt sich jedoch, dass eine gründlichere Planungsphase dazu hätte beitragen können, einige Probleme frühzeitig zu erkennen und zu vermeiden. So haben beispielsweise die differenzielle Datenübertragung und das Flashen der ATtinys und ATmegas viel Zeit in Anspruch genommen. Schwierigkeiten, die vermutlich durch eine bessere Abstimmung der Komponenten im Vorfeld hätten reduziert oder ganz vermieden werden können.

Des Weiteren wird in Anbetracht der noch verbleibenden Zeit das Audio-Modul aus dem Plan für dieses Projekt gestrichen und sich erstmal lediglich auf das Tastaturmodul fokussiert. Da dieses Projekt nach Abschluss des Faches \glqq Bussysteme und Sensorik\grqq{} für den privaten Gebrauch der Gruppenmitglieder weiterentwickelt wird, wird dieses Modul wahrscheinlich in Zukunft eingeführt. Das grobe Layout eines weiteren Moduls wird sich nicht grundlegend von dem des Tastaturmoduls unterscheiden, daher liegt die Priorität erstmal in der Optimierung des vorhandenen Moduls.


\subsection{Persöhnliches Fazit}
\subsubsection{Johanna Boettcher}

\subsubsection{Marcel Eßmann}
Das Projekt hat mir sehr viel Freude bereitet. Da wir alle von der Idee überzeugt waren ließ sich recht schnell ein Plan entwicklen. Die Probleme die wir hatten konnten wir meist mit gemeinsamer Denkleistung beheben oder alternativen finden.
Wir haben uns vermutlich etwas viel vorgenommen, da wir jetzt am Ende eigentliche in komplett fertiges Produkt haben und nutzen wollten. Dies hat jedoch dazu geführt, dass wir als Gruppe auch nach dem Hochschulmodul noch weiter an diesem Produkt arbeiten werden und zumindest ein für uns zufriedenstellendes Produkt entwickeln. 

\subsubsection{Peter Fischer}

\subsubsection{Jannik Wendt}